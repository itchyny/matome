\documentclass[a4j,10pt]{jarticle}
\def\defi#1#2#3{#1\quad$\displaystyle #2 \unit{[#3]}$}
\setlength{\topmargin}{0mm}
\setlength{\oddsidemargin}{5mm}
\setlength{\evensidemargin}{5mm}
\setlength{\textwidth}{200mm}
\setlength{\textheight}{304mm}
\setlength{\headsep}{0mm}
\setlength{\headheight}{0mm}
\setlength{\topskip}{0mm}
\setlength{\hoffset}{-1in}
\setlength{\voffset}{-0.5in}
\def\const{\mathrm{const}}
\def\uni#1{[\unit{#1}]}
\def\cell#1#2{#1\,\uni{#2}}
\def\dottedhole#1{\ar@{--}[#1]|!{[ld];[lu]}\hole}
\def\dottedholea#1{\ar@{--}[#1]|!{[ld];[lu]}\hole |!{+(-9.5,-10);+(-9.5,10)}\hole }
\def\dotted#1{\ar@{--}[#1]}
\def\dotteda#1{\ar@{--}[#1] |!{+(11.5,-50);+(11.5,50)}\hole}
\def\arrow#1#2#3{\ar[#1]|-(.4){\uni{#2}}_(.3){#3}}
\def\arrowq#1#2#3{\ar[#1]|-(.4){\uni{#2}}_(.35){#3}}
\def\arrowqq#1#2{\ar[#1]|-(.3){\uni{#2}}|-(.3)\hole|-(.5)\hole}
\def\arroww#1#2#3{\ar[#1]|-(.25){\uni{#2}}^(.4){#3}|!{[ll];[ru]}\hole}
\def\arrowu#1#2#3{\ar[#1]|-(.4){\uni{#2}}^(.4){#3}}
\def\arrowuu#1#2#3{\ar[#1]|-(.4){\uni{#2}}^(.5){#3}}
\def\arrowx#1#2#3{\ar[#1]|-(.6){\uni{#2}}^(.6){#3}}
\def\arrowy#1#2#3{\ar@/^3pc/[#1]|-(.55){\uni{#2}}^(.58){#3}}
\def\arrowyyyyy#1#2#3{\ar@/^3pc/[#1]|-(.5){\uni{#2}}^(.52){#3}}
\def\arrowyyy#1#2#3{\ar@/^3pc/[#1] |-(0.24)\hole |-(0.25){} |-(0.42)\hole|-(.55){\uni{#2}}^(.58){#3}|!{[luuu];[uuur]}\hole  |-(.766)\hole |-(.77){} |-(.835)\hole}
% |-(0.24)\hole |-(0.42)\hole だと, なぜかその間が切れてしまう. (0.2), (0.4)だと切れない. 謎い. これを回避するために, 空の(0.25){}を入れてる.
\def\arrowyyyy#1{\ar@{-}@/_3pc/[#1] |!{[ld];[rr]}\hole |!{[ld];[rd]}\hole |-(.45)\hole |!{[lddd];[dd]}\hole |!{[lddd];[rddd]}\hole}
\def\arrowyy#1#2#3#4#5{\ar@/_3pc/@{<->}[#1]|-(0.2){\uni{#4}}^(0.2){#5}|-(.75){\uni{#2}}_(.75){#3}}
\def\arrowz#1#2#3{\ar@/^2pc/[#1] |!{+(-10,-10);+(-10,10)}\hole|-(.7){\uni{#2}}_(.6){#3}}
% \def\arrowzz#1#2#3{\ar@/^2.2pc/[#1] |!{[ru];[rd]}\hole |!{+(71.3,-10);+(71.3,10)}\hole |!{[rr];[rrdd]}\hole |-(.8){\uni{#2}}_(.8){#3\quad}}
\def\arrowzz#1#2#3{\ar@/^2pc/[#1] |-(.2)\hole|-(.3)\hole |-(.545)\hole |-(.625)\hole|!{[rr];[rrdd]}\hole |-(.8){\uni{#2}}_(.8){#3\quad}}
%\def\arrowp#1#2#3{\ar[#1]|-(.5){\uni{#2}}_(.3){#3\quad}|!{[uu];[uuu]}\hole|!{[luuu];[ruuu]}\hole|!{[luuu];[rruuuu]}\hole}
%\def\arrowp#1#2#3{\ar[#1] _(.3){#3} |!{[ru];[lu]}\hole |-{\uni{#2}} |!{[luuu];[ruuu]}\hole |!{[luuu];[ruuuru]}\hole }
\def\arrowp#1#2#3{\ar@{->}[#1] _(.35){#3} |!{[u];[u]}\hole |-(0.4){\uni{#2}} |!{[uuu];[uuu]}\hole |!{[luuu];[ruuuru]}\hole} % これだと, luuu->ruuuruでの計算で, 座標が右のほうに行ってしまって最後が表示されない(推測) {-->}にすると表示されるのは謎い. 仕方ないので, Φのほうから下に降ろす線も描いてる
%\def\arrowp#1#2#3{\ar[#1] _(.3){#3} |\hole |-{\uni{#2}} }

\usepackage{bm}
\usepackage{booktabs}
\def\theorem#1#2{#1\quad$\displaystyle#2$}
\usepackage{prelude}
\usepackage[all]{xy}
\pagestyle{empty}
\def\B{\bm B}
\def\A{\bm A}
\def\D{\bm D}
\def\F{\bm F}
\def\E{\bm E}
\def\H{\bm H}
\def\J{\bm J}
\def\S{\bm S}
\def\P{\bm P}
\def\M{\bm M}
\def\r{\bm r}
\def\s{\bm s}
\def\v{\bm v}
\def\m{\bm m}
\begin{document}

\[\xymatrix@=0.70in{
  & \cell{\varPhi, q_m}{Wb}\dotted{rr}\ar @{-}[d]|!{[ld];[rr]}\hole \if0\arrowyyyy{dddd}\fi     &        & \cell{Q}{C}\\
 \cell V V\arrowuu{ur}{s}{e=-\fracpd\varPhi t}\arrowx{rrru}{F}{Q=CV}      &          & \cell I A\arrowu{ll}{O}{V=RI}\arroww{ur}{s}{I=\diff Q t}\arroww{lu}{H}{\varPhi=LI}        & \\
 & {\A}{} \arrowqq{uu}{m}  |!{[lu];[ruru]}\hole       & *=0{}       & \dottedholea{ll}\dotted{ld}\\
 \cell{\E}{Vm^{-1}}\arrow{uu}{m}{\E=\grad V}\dotted{rr} |!{+(20,-10);+(20,10)}\hole |!{+(32.5,-10);+(32.5,10)}\hole \dotted{ru}\arrowzz{rrrd}{Fm^{-1}}{\D=\varepsilon\E}\arrowyy{rrdd}{O^{-1}m^{-1}}{\J=\sigma\E}{Om}{\E=\rho\J } &          & \cell{\H,\M}{Am^{-1}}\arrow{uu}{m}{\oint_C\H\cdot\dd s=I_f}\arrowz{dl}{Hm^{-1}}{\B=\mu\H}       & \\
 & \cell{\B}{T}\arrowqq{uu}{m} \arrowyyy{uuuu}{m^2}{\varPhi=\int_S\B\cdot\dd\S} &        & \cell{\D,\P}{Cm^{-2}}\arrowq{uuuu}{m^2}{\oint_S\D\cdot\dd\bm{S}=Q}\dottedhole{ll}\dotted{ld}\\
 {{\rot\E=-\frac{\partial\B}{\partial t}} }\dotted{uu}\dotted{rr}\dotteda{ru}     &          & \cell{\J}{Am^{-2}}\arrow{uu}{m}{\rot\H=\J +\frac{\partial\D}{\partial t}}\arrowy{uuuu}{m^2}{I=\int_S\J\cdot\dd\S}\\
 & \cell{\rho_m}{Wbm^{-3}}\ar@{--}[uu]|!{[u];[u]}\hole |-(.73)\hole      &        & \cell{\rho}{Cm^{-3}}\arrow{uu}{m}{\diver\D=\rho}\arrowyyyyy{uuuuuu}{m^3}{Q=\int_v\rho\dd v}\dottedhole{ll}\dotted{ld}\\
 *=0{} \dotted{uu}\dotted{rr}\dotted{ru}           &          & \diver\J +\frac{\partial\rho}{\partial t}=0\dotted{uu} |!{+(-10,-10);(10,-10)}\hole      & \\
}\]



\def\cell#1#2{#1\,\unit{[#2]}}
\def\uni#1{}
\vspace{-10mm}
\def\tom{\ar@{->}[uu]|-{\unit{[m]}}}
\def\tos{\ar@{->}[ur]|-{\unit{[s]}}}
\def\toO{\ar@{->}[ll]|!{[]+(-20,-13);[]+(-20,13)}\hole|-{\unit{[O]}}}
\def\tomm{\ar[dd]|-{\uni{m^{-1}}}}
\def\toms{\ar@{->}+(-20,-13)*{\cell{f,\omega, \dd/\dd t}{s^{-1}}}|-{\uni{s^{-1}}}}
\def\tomO{\ar[rr]|-{\uni{O^{-1}}}}
\def\arrr{\dotted{rr}}
\def\ardd{\dotted{dd}}
\def\arddsplit{\dotted{dd}|!{[ld];[rr]}\hole|!{[ld];[rd]}\hole}
\def\aruu{\dotted{uu}}
\def\arll{\dotted{ll}}
\def\arllsplit{\dotted{ll}|!{[ld];[lu]}\hole}
\def\arllsplitp{\dotted{ll}|!{[]+(-20,-13);[]+(-20,13)}\hole}
\def\arrrsplit{\dotted{rr}|!{[rd];[ru]+(-20,-13)}\hole|!{[rd];[ru]}\hole}
\def\ard{\dotted{d}}
\def\arur{\dotted{ur}}
\def\ardr{\dotted{dr}}
\def\ardl{\dotted{dl} |!{+(-8,-30);+(-8,30)}\hole}
\def\arrowzp{\ar@/^2pc/[dl]|-(0.4){\uni{Hm^{-1}}}|!{[]+(-20,-13);[]+(-20,13)}\hole}
\def\arrowzzp{\ar@/_2pc/[rrrd]|-(0.25){\uni{Fm^{-1}}}|!{[rr];[rrdd]}\hole }
\def\tolv{\ar@{--}'+(-20,-13)*{\cell{v}{m s^{-1}}}'[dd]+(-20,-13)}
\def\tov{\ar@{->}[uu]+(-20,-13)+(3,-2)}
\def\toL{\ar@{->}[ul]|-{\uni{H,Os}}|!{+(-12,-12);+(-12,12)}\hole}
\def\toC{\ar@{->}[urrr]|-{\uni{F,O^{-1}s}}}
\def\arrowyyp{\ar@/_3pc/@{->}[rrdd]|-(0.7){\uni{O^{-1}m^{-1}}}}
\def\tof{\ar@{--}'+(-20,-13)'[uu]+(-20,-13) |!{+(-10,-8);+(10,-8)}\hole}
\def\toT{\ar@{--}'[ur]'[dr]}
\[\xymatrix@=0.50in{
  &&*=0{}& *=0{} &\\
      &      & \cell{x, \lambda,\r}{m} \tolv \toT &      &      &      & \\
      &      \cell{L}{H,Os} \arrrsplit\arddsplit      &      & \cell{t,T}{s}\arrr\arddsplit      &      & \cell{C}{F,O^{-1}s} \ardd      &      \\
 \cell{R}{O,kg\cdot m^2 s^{-3} A^{-2}}\ardd\arur      &      & *=0{}\arrowyyp\toL\toC\tov\tom\tos\toO\tomO\tomm\toms \arrowzzp \arrowzp      &      & \cell{G}{S,\Omega^{-1}}\arur\ardd      &      & \\
      &      \cell{\mu}{Hm^{-1}}      &      & \sqrt{\mu\varepsilon}\ardl\arll      &      & \cell{\varepsilon}{Fm^{-1}} \arllsplit      &      \\
 *=0{} \arur      &      & \cell{k,\dd/\dd x, \nabla}{m^{-1}} \arllsplitp\tof      &      & \cell{\sigma}{O^{-1}s^{-1}}\arll\arur      &      & \\
 }\]
% \if0
\vspace{20mm}
\def\const{\mathrm{const}}
\def\uni#1{[\unit{#1}]}
\begin{tabular}{cc}
\toprule
静電界                                                         & 静磁界 \\ \hline\hline
真空中                                                         & 真空中 \\ \hline
\defi{真空の誘電率}{{\varepsilon_0}=8.854\times 10^{-12}}{Fm^{-1}}                               & \defi{真空の透磁率}{\mu_0={4\pi}/{10^7}}{Hm^{-1}}\\

\theorem{クーロンの法則}{\F = \frac 1{4\pi{\varepsilon_0}}\frac{\r_0 - \r_1}{{\abs{\r_0 - \r_1}}^3}Q_0Q_1}      & \theorem{磁極のクーロンの法則}{\F = \frac 1{4\pi\mu_0}\frac{\r_0 - \r_1}{{\abs{\r_0 - \r_1}}^3}q_{m0}q_{m1}}\\

\theorem{ガウスの法則}{\diver \E = \frac \rho {\varepsilon_0} \quad \oint_S \E \cdot \dd \bm{S}= \frac Q {\varepsilon_0}} & \theorem{磁極の定義より}{\diver \H = \frac{\rho_m}{\mu_0}\quad \oint_S \H \cdot \dd \S = \frac{q_m}{\mu_0}}\\
                                                            & \theorem{磁束保存}{\diver \B = 0 \text{(Maxwell 4)}\quad \oint_S \B \cdot \dd \S = 0}\\

\theorem{電界保存}{\rot \E = 0 (\B = \const)}                                         & \theorem{真電流がないなら}{\rot \H = 0 (\J = 0, \D = \const)}\\


\defi{電界}{\E}{NC^{-1},Vm^{-1}}                                                        & \defi{磁界の強さ}{\H}{Am^{-1}}\\
\defi{電荷密度}{\rho=\lim_{\delta v \to 0}\frac{\delta Q}{\delta v}}{Cm^{-3}}                                             & \defi{磁束密度}{\B}{T, Wbm^{-2}}\\
\defi{電荷}{Q=\int_v \rho \dd v}{C}                                                        & \theorem{ビオ・サバールの法則}{\delta \B = \frac{\mu_0}{4 \pi}\frac{I \dd \s \times \r}{r^3}}\\
\theorem{電位}{V = - \int _{\infty}^{p}\E \cdot \dd \S \unit{V}\quad \E = -\grad V}                            & \theorem{アンペールの法則}{\oint_C \B\cdot \dd \s = \mu_0 I \quad \rot \B = \mu_0 \J}\\
\theorem{ポアソン方程式}{\nabla ^ 2 V = - \frac{\rho}{{\varepsilon_0}}}                           & \theorem{ローレンツ力}{\F = q (\E + \v \times \B)}\\
\defi{静電容量}{C= Q / V}{F,CV^{-1}}                                                         & \\ \hline\hline

誘電体                                                         & 磁性体 \\ \hline
                                                            & \defi{磁気モーメントの強さ}{\m = I \Delta \S}{Am^2}\\
\defi{分極の強さ}{\P=\rho_0 \delta \r}{Cm^{-2}}                                         & \defi{磁化の強さ}{\M = \Delta \m / \Delta v}{Am^{-1}}\\
\defi{分極電荷の体積密度}{\rho_P=-\diver \P}{Cm^{-3}}                                          & \defi{磁極の強さの体積密度}{\rho_m=-\diver (\mu_0 \M)}{Wbm^{-3}}\\
\defi{分極電荷}{Q_P=\int_v \rho_P \dd v = -\oint_S \P \cdot \dd \S}{C,FV}                                           & \defi{磁極の強さ}{q_m = \int_v \rho_m \dd v}{Wb,Tm^2}\\
                                                            & \defi{磁界の強さ}{\H = \frac{\B}{\mu_0}- \M}{Am^{-1}}\\
                                                            & \defi{磁気分極}{\bm{J_m}= \mu_0 \M}{T}\\
\defi{電束密度}{\D = {\varepsilon_0} \E + \P}{C m^{-2}}                               & $\B = \mu_0 \H + \bm{J_m}$ \\



\theorem{ガウスの法則}{\diver \D = \rho \text{(Maxwell 3)}\quad \oint_S \D \cdot \dd \bm{S}= Q}                       & \theorem{アンペールの法則}{\oint_C \H\cdot \dd s = I_f \quad \rot \H = \bm{J_f}}\\


\hline
\theorem{等方性誘電体}{\P = \chi \E = \chi_S {\varepsilon_0} \E}                        & \theorem{等方性磁性体}{\M = \chi \H}\\

\defi{分極率}{\chi}{C^2N^{-1}m^{-2}}\quad \defi{比分極率}{\chi_s}{-}                            & \defi{磁化率}{\chi}{-}\\


$\D = \varepsilon \E = {\varepsilon_0} \varepsilon_s \E$                          & $\B = \mu \H = \mu_s \mu_0 \H$ \\
\defi{誘電体の誘電率}{\varepsilon}{Fm^{-1}}\quad \defi{比誘電率}{\varepsilon_s}{-}                                               & \defi{磁性体の透磁率}{\mu}{Hm^{-1}}\quad \defi{比透磁率}{\mu_s}{-}\\


\theorem{電界のエネルギー密度}{\frac 1 2 \E \cdot \D}                                        & \theorem{磁界のエネルギー密度}{\frac 1 2 \H \cdot \B}\\
%\theorem{単位面積当たりの力}{f=\frac 1 2 \E \cdot \D}                                      & \\
\midrule\midrule

定常電流界                                                       & 磁気回路 \\ \hline
\theorem{電界は保存的(KVL)}{\rot \E = 0}                                                   & \theorem{真電流$\J$がないなら}{\rot \H = 0}\\
\defi{電流}{I = \diff Q t = \int_S \J \cdot \dd \S}{A,Cs^{-1}}                       & \defi{磁束}{\varPhi = \int_S \B \cdot \dd \S}{Wb, Tm^2}\\
\theorem{電流連続}{\diver \J + \frac{\partial \rho}{\partial t}=0 \to \diver \J = 0 \text{(KCL)}}                   & \theorem{磁束保存}{\diver \B = 0}\\
\theorem{オームの法則}{V = RI \quad \J = \sigma \E = \frac{\E}\rho}                                                & \theorem{}{NI=R_m\varPhi}\quad\theorem{}{\B = \mu \H = \frac{\H}{\nu}}\\
\defi{抵抗}{R=\frac{\rho l}{S}=\frac l{\sigma S}}{O,VA^{-1}}                               & \defi{磁気抵抗}{R_m=\frac l{\mu S}}{A/Wb}\\
\defi{導電率}{\sigma}{O^{-1}m^{-1}}\quad \defi{抵抗率}{\rho}{Om}                               & \defi{透磁率}{\mu}{Hm^{-1}}\quad \defi{磁気抵抗率}{\nu}{H^{-1}m}\\
\theorem{起電力(ファラデーの電磁誘導の法則)}{e = \oint_C \E \cdot \dd \s = -\frac{\partial \varPhi}{\partial t}\unit{[V]}}         & \theorem{起磁力}{NI = \oint_C \H \cdot \dd \s\unit{[A]}}\\
\hline\hline
\multicolumn{2}{c}{マックスウェル方程式}\\
\theorem{ファラデーの電磁誘導の法則}{\rot \E = - \frac{\partial \B}{\partial t}}                                            & \theorem{アンペール+変位電流}{\rot \H = \J + \frac{\partial \D}{\partial t}}\\
\theorem{ガウスの法則}{\diver \D = \rho}                                                    & \theorem{磁束保存}{\diver \B = 0}\\
一様なら$\D = \varepsilon \E$, $\B = \mu \H$                                                & 一様なら$\J = \sigma \E$ \\
\bottomrule
\end{tabular}
% \fi
\end{document}


