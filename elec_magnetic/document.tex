\documentclass[a4j,10pt]{jarticle}
\def\defi#1#2#3{#1\quad$\displaystyle #2 \unit{[#3]}$}
\setlength{\hoffset}{-0.80in}
\setlength{\voffset}{-0.7in}
\def\const{\mathrm{const}}
\def\uni#1{[\unit{#1}]}

\usepackage{bm}
\def\theorem#1#2{#1\quad$\displaystyle#2$}
\usepackage{booktabs}
\usepackage{prelude}
\pagestyle{empty}
\def\B{\bm B}
\def\D{\bm D}
\def\F{\bm F}
\def\E{\bm E}
\def\H{\bm H}
\def\J{\bm J}
\def\S{\bm S}
\def\P{\bm P}
\def\M{\bm M}
\def\r{\bm r}
\def\s{\bm s}
\def\v{\bm v}
\def\m{\bm m}
\begin{document}
\begin{table}[htbp]
\begin{tabular}{cc}
\toprule
静電界                                                         & 静磁界 \\ \hline\hline
真空中                                                         & 真空中 \\ \hline
\defi{真空の誘電率}{\varepsilon_0=8.854\times 10^{-12}}{Fm^{-1}}                               & \defi{真空の透磁率}{\mu_0={4\pi}/{10^7}}{Hm^{-1}}\\

\theorem{クーロンの法則}{\F = \frac 1{4\pi\varepsilon_0}\frac{\r_0 - \r_1}{{\abs{\r_0 - \r_1}}^3}Q_0Q_1}      & \theorem{磁極のクーロンの法則}{\F = \frac 1{4\pi\mu_0}\frac{\r_0 - \r_1}{{\abs{\r_0 - \r_1}}^3}q_{m0}q_{m1}}\\

\theorem{ガウスの法則}{\diver \E = \frac \rho \varepsilon_0 \quad \oint_S \E \cdot \dd \bm{S}= \frac Q \varepsilon_0} & \theorem{磁極の定義より}{\diver \H = \frac{\rho_m}{\mu_0}\quad \oint_S \H \cdot \dd \S = \frac{q_m}{\mu_0}}\\
                                                            & \theorem{磁束保存}{\diver \B = 0 \text{(Maxwell 4)}\quad \oint_S \B \cdot \dd \S = 0}\\

\theorem{電界保存}{\curl \E = 0 (\B = \const)}                                         & \theorem{真電流がないなら}{\curl \H = 0 (\J = 0, \D = \const)}\\


\defi{電界}{\E}{NC^{-1},Vm^{-1}}                                                        & \defi{磁界の強さ}{\H}{Am^{-1}}\\
\defi{電荷密度}{\rho=\lim_{\delta v \to 0}\frac{\delta Q}{\delta v}}{Cm^{-3}}                                             & \defi{磁束密度}{\B}{T, Wbm^{-2}}\\
\defi{電荷}{Q=\int_v \rho \dd v}{C}                                                        & \theorem{ビオ・サバールの法則}{\delta \B = \frac{\mu_0}{4 \pi}\frac{I \dd \s \times \r}{r^3}}\\
\theorem{電位}{V = - \int _{\infty}^{p}\E \cdot \dd \S \unit{V}\quad \E = -\grad V}                            & \theorem{アンペールの法則}{\oint_C \B\cdot \dd \s = \mu_0 I \quad \curl \B = \mu_0 \J}\\
\theorem{ポアソン方程式}{\nabla ^ 2 V = - \frac{\rho}{\varepsilon_0}}                           & \theorem{ローレンツ力}{\F = q (\E + \v \times \B)}\\
\defi{静電容量}{C= Q / V}{F,CV^{-1}}                                                         & \\ \hline\hline

誘電体                                                         & 磁性体 \\ \hline
                                                            & \defi{磁気モーメントの強さ}{\m = I \Delta \S}{Am^2}\\
\defi{分極の強さ}{\P=\rho_0 \delta \r}{Cm^{-2}}                                         & \defi{磁化の強さ}{\M = \Delta \m / \Delta v}{Am^{-1}}\\
\defi{分極電荷の体積密度}{\rho_P=-\diver \P}{Cm^{-3}}                                          & \defi{磁極の強さの体積密度}{\rho_m=-\diver (\mu_0 \M)}{Wbm^{-3}}\\
\defi{分極電荷}{Q_P=\int_v \rho_P \dd v = -\oint_S \P \cdot \dd \S}{C,FV}                                           & \defi{磁極の強さ}{q_m = \int_v \rho_m \dd v}{Wb,Tm^2}\\
                                                            & \defi{磁界の強さ}{\H = \frac{\B}{\mu_0}- \M}{Am^{-1}}\\
                                                            & \defi{磁気分極}{\bm{J_m}= \mu_0 \M}{T}\\
\defi{電束密度}{\D = \varepsilon_0 \E + \P}{C m^{-2}}                               & $\B = \mu_0 \H + \bm{J_m}$ \\



\theorem{ガウスの法則}{\diver \D = \rho \text{(Maxwell 3)}\quad \oint_S \D \cdot \dd \bm{S}= Q}                       & \theorem{アンペールの法則}{\oint_C \H\cdot \dd s = I_f \quad \curl \H = \bm{J_f}}\\


\hline
\theorem{等方性誘電体}{\P = \chi \E = \chi_S \varepsilon_0 \E}                        & \theorem{等方性磁性体}{\M = \chi \H}\\

\defi{分極率}{\chi}{C^2N^{-1}m^{-2}}\quad \defi{比分極率}{\chi_s}{-}                            & \defi{磁化率}{\chi}{-}\\


$\D = \varepsilon \E = \varepsilon_0 \varepsilon_s \E$                          & $\B = \mu \H = \mu_s \mu_0 \H$ \\
\defi{誘電体の誘電率}{\varepsilon}{Fm^{-1}}\quad \defi{比誘電率}{\varepsilon_s}{-}                                               & \defi{磁性体の透磁率}{\mu}{Hm^{-1}}\quad \defi{比透磁率}{\mu_s}{-}\\


\theorem{電界のエネルギー密度}{\frac 1 2 \E \cdot \D}                                        & \theorem{磁界のエネルギー密度}{\frac 1 2 \H \cdot \B}\\
%\theorem{単位面積当たりの力}{f=\frac 1 2 \E \cdot \D}                                      & \\
\midrule\midrule

定常電流界                                                       & 磁気回路 \\ \hline
\theorem{電界は保存的(KVL)}{\curl \E = 0}                                                   & \theorem{真電流$\J$がないなら}{\curl \H = 0}\\
\defi{電流}{I = \diff Q t = \int_S \J \cdot \dd \S}{A,Cs^{-1}}                       & \defi{磁束}{\varPhi = \int_S \B \cdot \dd \S}{Wb, Tm^2}\\
\theorem{電流連続}{\diver \J + \frac{\partial \rho}{\partial t}=0 \to \diver \J = 0 \text{(KCL)}}                   & \theorem{磁束保存}{\diver \B = 0}\\
\theorem{オームの法則}{V = RI \quad \J = \sigma \E = \frac{\E}\rho}                                                & \theorem{}{NI=R_m\varPhi}\quad\theorem{}{\B = \mu \H = \frac{\H}{\nu}}\\
\defi{抵抗}{R=\frac{\rho l}{S}=\frac l{\sigma S}}{O,VA^{-1}}                               & \defi{磁気抵抗}{R_m=\frac l{\mu S}}{A/Wb}\\
\defi{導電率}{\sigma}{O^{-1}m^{-1}}\quad \defi{抵抗率}{\rho}{Om}                               & \defi{透磁率}{\mu}{Hm^{-1}}\quad \defi{磁気抵抗率}{\nu}{H^{-1}m}\\
\theorem{起電力(ファラデーの電磁誘導の法則)}{e = \oint_C \E \cdot \dd \s = -\frac{\partial \varPhi}{\partial t}\unit{[V]}}         & \theorem{起磁力}{NI = \oint_C \H \cdot \dd \s\unit{[A]}}\\
\hline\hline
\multicolumn{2}{c}{マックスウェル方程式}\\
\theorem{ファラデーの電磁誘導の法則}{\curl \E = - \frac{\partial \B}{\partial t}}                                            & \theorem{アンペール+変位電流}{\curl \H = \J + \frac{\partial \D}{\partial t}}\\
\theorem{ガウスの法則}{\diver \D = \rho}                                                    & \theorem{磁束保存}{\diver \B = 0}\\
一様なら$\D = \varepsilon \E$, $\B = \mu \H$                                                & 一様なら$\J = \sigma \E$ \\
\bottomrule
\end{tabular}
\end{table}
\end{document}
